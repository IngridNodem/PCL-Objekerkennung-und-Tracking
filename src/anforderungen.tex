% !TeX program = pdflatex
\documentclass[12pt,a4paper]{article}

% Packages
\usepackage[T1]{fontenc}
\usepackage[utf8]{inputenc}
\usepackage[ngerman]{babel}
\usepackage{lmodern}
\usepackage{geometry}
\usepackage{setspace}
\usepackage{hyperref}
\usepackage{tabularx}
\usepackage{csquotes}
\usepackage[backend=biber,style=authoryear,sorting=nyt,maxbibnames=99]{biblatex}
\addbibresource{literatur.bib}

\geometry{left=25mm,right=20mm,top=20mm,bottom=25mm}
\onehalfspacing
\setlength{\parindent}{0pt}
\setlength{\parskip}{6pt}
\newcolumntype{Y}{>{\raggedright\arraybackslash}X}

\begin{document}

\section*{Anforderungen an den Algorithmus}

Die Punktwolken des Ouster~OS1 bilden die Grundlage einer Verarbeitungskette, die relevante Objekte im Umfeld extrahiert und nachverfolgt. Die hohe Winkelauflösung des Sensors ermöglicht eine präzise Rekonstruktion lokaler Geometrien \parencite{OusterOS1}, erfordert jedoch eine robuste Vorverarbeitung: Boden- und Störanteile müssen entfernt, die Punktdichte kontrolliert reduziert und gleichzeitig ausreichende Detailinformationen für verlässliche Bounding Boxes bewahrt werden.

Diese Schritte folgen den Empfehlungen der Fahrzeugautomatisierungs-Literatur, die deterministische Filterung und kontrollierte Latenz als Beitrag zur funktionalen Sicherheit hervorhebt \parencite{Arnold2019Survey,Macenski2022ROS2}. Eine Bodenentfernung senkt die Falsch-Positiv-Rate und verbessert die Selektivität beim Clustering \parencite{gomes2023survey}; konfigurierbare Parameter erleichtern die Adaption an unterschiedliche Szenarien wie dichten Verkehr, freie Flächen oder geneigte Fahrbahnen.

Um den Einsatz auf Standardhardware zu ermöglichen und die Reproduzierbarkeit sicherzustellen, wird eine End-to-End-Latenz unter \(100\,\text{ms}\) bei \(10\,\text{Hz}\) angestrebt, verbunden mit moderater CPU/RAM-Auslastung und klar definierten ROS~2-Nachrichten für maschinen- wie terminallesbare Ausgaben. Die Einbindung in die bestehende GUI des Projekts \emph{Carception~X} (IFZN) stellt darüber hinaus Kompatibilität mit den vorhandenen Betriebsmodi (Sensor, Sensor+PC, Echtzeitbetrieb) sicher.\footnote{\url{https://www.th-nuernberg.de/einrichtungen-gesamt/in-institute/institut-fuer-fahrzeugtechnik/projekte/}}

Die detaillierten Anforderungen und dazugehörigen Messkriterien sind in Tabelle~\ref{tab:anforderungen_messkette} zusammengefasst.

\begin{table}[h]
\centering
\begin{tabularx}{\textwidth}{|c|X|X|}
\hline
\textbf{Nr.} & \textbf{Anforderung} & \textbf{Messkriterium} \\
\hline

1 &
OS1-Punktwolken werden empfangen &
Topic \texttt{/ouster/points}; Rate $\geq 10$ Hz \\[0.15cm]
\hline

2 &
Objektliste mit Pose und Abmessungen &
Topic \texttt{/tracks\_raw} (\texttt{vision\_msgs/Detection3DArray}) \\[0.15cm]
\hline

3 &
Bodenpunkte und Störsignale entfernen &
Topic \texttt{/obstacle\_points} ohne Bodenanteile \\[0.15cm]
\hline

4 &
Reduzierte Punktdichte ohne wesentlichen Geometrieverlust &
Konfigurierbare Voxelgröße; stabile FPS \\[0.15cm]
\hline

5 &
Laufzeitparameter anpassbar &
Parameter via Launch setzbar \\[0.15cm]
\hline

6 &
End-to-End-Latenz <100 ms bei 10 Hz &
Zeitstempel-Differenz Ein-/Ausgang; stabile 10 Hz \\[0.15cm]
\hline

7 &
Stabiler Betrieb auf Standardhardware &
CPU/RAM-Auslastung $\leq 60\,\%$ bei stabilen $10\,\mathrm{Hz}$ \\[0.15cm]
\hline

8 &
Maschinen- und terminallesbare Ausgabe &
Definierte ROS-Nachrichten + kompakte Konsole \\[0.15cm]
\hline

9 &
Integration in bestehende GUI &
Algorithmus ausführbar bei Moden Sensor, Sensor+PC und Echtzeitanwendung \\[0.05cm]
\hline

\end{tabularx}
\caption{Anforderungen an die Verarbeitungsstrecke mit zugehörigen Messkriterien}
\label{tab:anforderungen_messkette}
\end{table}

\printbibliography

\end{document}
