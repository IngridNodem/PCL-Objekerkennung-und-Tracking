\subsubsection{Gerade Strecke, Stand (1\,Minute, 0\,km/h)}
\label{subsec:gerade_strecke_stand}

Ziel dieses Tests ist es, das Verhalten der Verarbeitungskette bei einer statischen Szene zu bewerten und dabei insbesondere Sensorstabilität, Grundlatenz und Grundfrequenz zu untersuchen. Der Versuch wurde auf einer geraden Strecke im Stand (\(0\,\mathrm{km/h}\)) mit einer Dauer von \(60\,\mathrm{s}\) durchgeführt.

\paragraph*{Detektionen}
Während des Tests wurden über die Logausgaben des \texttt{tracking\_node} insgesamt
\(\,18\,\)~Detektionen der Klasse \emph{Pkw} und \(\,4\,\)~Detektionen der Klasse \emph{Fahrradfahrer} registriert, d.\,h. insgesamt \(\,22\,\)~Tracks in einer weitgehend statischen Szene. Dies deutet darauf hin, dass der Tracker bei nahezu unbewegten Objekten vergleichsweise häufig neue IDs vergibt. Dies wird durch die Auswertung des \texttt{ids\_probe} bestätigt: die Geburtsraten liegen typischerweise zwischen ca.~\(24\) und \(74\)~Tracks pro Minute, bei gleichzeitig kurzen medianen Lebensdauern (\(t_\text{med} \approx 0{,}3{-}2{,}4\,\mathrm{s}\)). In den späteren Intervallen sinkt die Geburtsrate jedoch zeitweise auf \(0\)~Tracks pro Minute, während die mediane aktive Lebensdauer der wenigen verbleibenden IDs auf bis zu \(\approx 9{,}7\,\mathrm{s}\) anwächst.

\paragraph*{Ressourcenauslastung}
Die mittlere CPU- und Speicherauslastung der einzelnen Knoten wurde mit dem Werkzeug \texttt{resource\_probe.py} über die Testdauer von \(60\,\mathrm{s}\) erfasst. Tabelle~\ref{tab:stand_resource_usage} fasst die Ergebnisse zusammen.

\begin{table}[H]
  \centering
  \caption{Ressourcenauslastung der Verarbeitungskette (Gerade Strecke, Stand, 60\,s)}
  \label{tab:stand_resource_usage}
  \begin{tabular}{lrr}
    \toprule
    \textbf{Knoten} & \textbf{avg CPU [\%]} & \textbf{avg RSS [MB]} \\
    \midrule
    \texttt{crop\_box\_node}           & 11.60 &  38.24 \\
    \texttt{voxel\_filter\_node}       &  3.54 &  27.19 \\
    \texttt{ransac\_ground\_node}      &  6.44 &  53.66 \\
    \texttt{cluster\_extraction\_node} &  8.39 &  52.94 \\
    \texttt{tracking\_node}            &  1.90 &  24.57 \\
    \midrule
    \textbf{Gesamt}                    & 31.87 & 196.60 \\
    \bottomrule
  \end{tabular}
\end{table}

Die Verarbeitungskette nutzt im Mittel nur rund \(32\,\%\) der verfügbaren CPU-Leistung und etwa \(200\,\mathrm{MB}\) Hauptspeicher. Damit verbleibt ein deutlicher Sicherheitsabstand zu typischen Echtzeitanforderungen auf dem verwendeten Versuchslaptop.

\paragraph*{Frequenzen und Frame-Loss}
Die Eingangs- und Ausgangsfrequenzen sowie mögliche Frame-Verluste wurden mit dem Werkzeug \texttt{frame\_loss\_probe.py} und den ROS-2-Metriken (\texttt{ros2 topic hz}) untersucht. Für den Sensor-Input \texttt{/ouster/points} ergibt sich eine mittlere Rate von etwa
\(\,2{,}5\,\mathrm{Hz}\), während das Ausgabe-Topic \texttt{/tracks\_raw} mit ca.
\(\,1{,}8\,\mathrm{Hz}\) arbeitet. Die \emph{pass\%}-Werte entlang der Verarbeitungskette (\texttt{crop\_box} \(\rightarrow\) \texttt{voxel\_filter} \(\rightarrow\) \texttt{ransac\_ground} \(\rightarrow\) \texttt{cluster\_extraction} \(\rightarrow\) \texttt{tracking}) liegen überwiegend im Bereich von \(80{-}100\,\%\). Die gemeldeten \emph{gaps} betragen in den meisten Intervallen \(0\); nur vereinzelt treten Lücken von einem Frame auf. Lediglich in den letzten Messfenstern fällt \texttt{crop\_box} infolge des Testendes kurzzeitig auf \(0\,\mathrm{Hz}\), wodurch \emph{pass\%} formal auf \(0\,\%\) absinken.

\paragraph*{Latenz}
Die End-to-End-Latenzen wurden mit dem Werkzeug \texttt{latency\_probe.py} anhand der Zeitdifferenzen zwischen den Topics \texttt{/points\_cropped}, \texttt{/points\_voxel}, \texttt{/obstacle\_points}, \texttt{/detections\_raw} und \texttt{/tracks\_raw} bestimmt. Typische mittlere Latenzwerte liegen bei
\begin{itemize}[noitemsep]
  \item \(\text{crop} \rightarrow \text{voxel}\): ca.~\(32{-}38\,\mathrm{ms}\),
  \item \(\text{voxel} \rightarrow \text{ransac}\): ca.~\(22\,\mathrm{ms}\),
  \item \(\text{ransac} \rightarrow \text{cluster}\): ca.~\(2{-}2{,}5\,\mathrm{ms}\),
  \item \(\text{cluster} \rightarrow \text{track}\): ca.~\(27{-}30\,\mathrm{ms}\).
\end{itemize}
Die daraus resultierende End-to-End-Latenz \(\text{crop} \rightarrow \text{track}\) beträgt im Mittel etwa \(86\,\mathrm{ms}\) und stabilisiert sich gegen Ende der Messung bei rund \(81\,\mathrm{ms}\). Damit wird die geforderte Grenze von \(100\,\mathrm{ms}\) klar unterschritten.

\paragraph*{Bewertung}
Zusammenfassend zeigt der Test „Gerade Strecke, Stand“, dass die Verarbeitungskette unter statischen Bedingungen stabil arbeitet, keine nennenswerten Frame-Verluste aufweist und sowohl Ressourcenverbrauch als auch End-to-End-Latenz im geforderten Bereich liegen. Auffällig ist lediglich die relativ hohe Anzahl an Track-Geburten bei kurzen Lebensdauern in einer im Wesentlichen konstanten Szene, was auf Optimierungspotential bei den Tracking-Parametern (z.\,B. Lebensdauer von Tracks, Positionsschwellen) hinweist, jedoch die Erfüllung der Echtzeitanforderungen nicht beeinträchtigt.

\subsubsection{Kurve (Urban, ca.\ 40\,km/h)}

Ziel dieses Tests war die Überprüfung, ob die Verarbeitungskette auch bei Fahrt mit ca.\ 40\,km/h in einer innerstädtischen Kurve eine stabile Datenrate und Latenz ohne längere Aussetzer erreicht.

\paragraph*{Ressourcenauslastung}
Die mit \texttt{resource\_probe.py} gemessene mittlere CPU-Auslastung der Verarbeitungsnodes liegt bei insgesamt
\(\approx 31{,}4\,\%\). Der \texttt{crop\_box\_node} weist mit rund \(17{,}9\,\%\) den höchsten Anteil auf, während
\texttt{voxel\_filter\_node}, \texttt{ransac\_ground\_node}, \texttt{cluster\_extraction\_node} und
\texttt{tracking\_node} jeweils unter \(5\,\%\) bleiben. Der durchschnittliche Hauptspeicherbedarf der Kette beträgt
\(\approx 198\,\text{MB}\). Damit liegen CPU- und Speicherlast im unkritischen Bereich, sodass auch bei höheren
Umgebungsanforderungen noch Reserven bestehen.

\paragraph*{Latenzverhalten}
Das Latenztool \texttt{latency\_probe} ermittelt für die einzelnen Verarbeitungsschritte die folgenden mittleren
Laufzeiten:
\begin{itemize}
  \item \texttt{crop\_box\_node} $\rightarrow$ \texttt{voxel\_filter\_node}: \(14{,}05\,\text{ms}\) (\(n = 146\)),
  \item \texttt{voxel\_filter\_node} $\rightarrow$ \texttt{ransac\_ground\_node}: \(7{,}36\,\text{ms}\) (\(n = 146\)),
  \item \texttt{ransac\_ground\_node} $\rightarrow$ \texttt{cluster\_extraction\_node}: \(3{,}06\,\text{ms}\) (\(n = 145\)),
  \item \texttt{cluster\_extraction\_node} $\rightarrow$ \texttt{tracking\_node}: \(14{,}81\,\text{ms}\) (\(n = 145\)).
\end{itemize}
Für die gesamte Verarbeitungskette vom Eingang der zugeschnittenen Punktwolke bis zur Ausgabe der Tracks ergibt sich
damit eine mittlere End-to-End-Latenz von
\[
  t_{\text{crop}\rightarrow\text{track}} \approx 39{,}14\,\text{ms} \quad (n = 145),
\]
die deutlich unter einem typischen Zielwert von \(100\,\text{ms}\) liegt. Die Verarbeitung kann somit auch im
Fahrszenario \emph{Kurve} als echtzeitfähig eingestuft werden.

\paragraph*{Datenrate und Aussetzer}
Die mittlere Verarbeitungsrate der Punktwolken bzw.\ Zwischenstufen beträgt im Test
\(\approx 3{,}7\,\text{Hz}\). Die gemessenen Perioden liegen zwischen \(0{,}154\,\text{s}\) und \(0{,}793\,\text{s}\)
bei einer Standardabweichung von \(\approx 0{,}095\,\text{s}\), was auf einen überwiegend gleichmäßigen Ablauf ohne
längere Aussetzer hinweist.

Für das Track-Topic (\emph{``tracks raw''}) ergibt sich eine mittlere Ausgaberate von etwa \(1{,}8\,\text{Hz}\).
Die Perioden liegen überwiegend im Bereich von \(\approx 0{,}3\,\text{s}\), erreichen vereinzelt jedoch Werte bis
\(1{,}44\,\text{s}\). Insgesamt bleibt die Track-Ausgabe damit hinreichend stabil, zeigt aber im Vergleich zur
Verarbeitungsrate der Punktwolken gelegentliche längere Intervalle, die auf interne Pufferungen oder
Verarbeitungsjitter im Tracking hinweisen können.

\subsubsection{Innenstadtfahrt mit Steigung (3~min, 30~km/h)}

Ziel dieses Versuchs war es, die Stabilität von Latenz und Datenrate unter innerstädtischen Bedingungen mit Steigungsanteil zu überprüfen und längere Aussetzer der Verarbeitungskette auszuschließen. Die mit dem \texttt{latency\_probe} aufgezeichneten Messwerte zeigen für die einzelnen Verarbeitungsstufen mittlere Latenzen von etwa
\(21{,}1\,\text{ms}\) (\texttt{crop\_box\_node}~\(\rightarrow\)~\texttt{voxel\_filter\_node}),
\(2{,}7\,\text{ms}\) (\texttt{voxel\_filter\_node}~\(\rightarrow\)~\texttt{ransac\_ground\_node}),
\(0{,}23\,\text{ms}\) (\texttt{ransac\_ground\_node}~\(\rightarrow\)~\texttt{cluster\_extraction\_node}) und
\(6{,}0\,\text{ms}\) (\texttt{cluster\_extraction\_node}~\(\rightarrow\)~\texttt{tracking\_node}).
Die End-to-End-Latenz von \texttt{crop\_box\_node} zu \texttt{tracking\_node} liegt im Mittel bei rund
\(30\,\text{ms}\) und nimmt während der Messung leicht ab, sodass der geforderte Grenzwert von
\(100\,\text{ms}\) mit deutlicher Reserve eingehalten wird.

Der Ressourcenverbrauch der Kette bleibt mit einer durchschnittlichen CPU-Last von
\(\approx 12{,}3\,\%\) und einem Gesamtspeicherbedarf von rund \(200\,\text{MB}\) auf dem Versuchsrechner moderat. Keiner der Nodes sticht als Engpass hervor; die Last verteilt sich im Wesentlichen auf
\texttt{crop\_box\_node}, \texttt{ransac\_ground\_node} und \texttt{cluster\_extraction\_node}.

Die mit \texttt{ros2 topic hz} gemessene Frequenz der Verarbeitungsdaten (z.\,B.\ \texttt{/detections\_raw}) liegt stabil bei etwa \(3{,}3\) bis \(3{,}4\,\text{Hz}\) mit geringen Schwankungen
(Intervalldauer \(\approx 0{,}3\,\text{s}\), maximale beobachtete Lücke \(\approx 0{,}36\,\text{s}\)).
Das Tracking-Topic erreicht nach der Initialisierungsphase eine effektive Aktualisierungsrate von
rund \(2\,\text{Hz}\). Ein in den Messdaten ausgewiesener maximaler Abstand von
\(55{,}5\,\text{s}\) ist als einzelner Start-Ausreißer zu interpretieren und deutet nicht auf wiederkehrende Aussetzer im laufenden Betrieb hin. Insgesamt erfüllt die Verarbeitungskette damit das Ziel einer stabilen Latenz und Datenrate ohne längere Unterbrechungen auch unter innerstädtischen Fahrbedingungen.

\section*{Ergebnisse – Kreuzung (50\,km/h, 3\,min, Bodenwellen/Steigung)}

Im Szenario \emph{Kreuzung} mit einer Fahrgeschwindigkeit von \(50\,\text{km/h}\) wurde das Verhalten der Verarbeitungskette unter kombinierter Rotation, Fahrbahnunebenheiten und wechselnder Objektgeometrie untersucht. Die Verarbeitungsgeschwindigkeit zeigt über die gesamte Versuchsdauer eine konstante Frequenz von rund \(3.21{-}3.24\,\text{Hz}\). Gleichzeitig nimmt die Standardabweichung der Intervallzeiten kontinuierlich ab (von ca.\ \(0.072\,\text{s}\) auf \(0.057\,\text{s}\)), was auf eine zunehmende Stabilisierung der Pipeline trotz dynamischer Fahrsituation hinweist.

Die gemessenen Latenzen der einzelnen Verarbeitungsschritte bleiben ebenfalls über den gesamten Versuch sehr stabil. Die Hauptanteile entfallen erwartungsgemäß auf die Kombination aus \texttt{CropBox} und \texttt{VoxelGrid}, die im Bereich von \(15.4{-}17.3\,\text{ms}\) liegen. Die folgenden Schritte --- RANSAC-Bodensegmentierung und Cluster-Extraktion --- weisen jeweils sehr geringe Laufzeiten auf (unter \(2\,\text{ms}\) bzw.\ \(0.5\,\text{ms}\)). Die resultierende End-zu-End-Latenz zwischen Eingangspunktwolke und Tracking-Output beträgt konstant \(21{-}24\,\text{ms}\), was für die getestete Geschwindigkeit ein robustes Reaktionsverhalten sicherstellt.

Bei den Tracking-Outputs (\texttt{tracks\_raw}) ergibt die Auswertung der Frequenzmessungen einen mittleren Wert von \(\approx 3{,}21\,\text{Hz}\) (Mittel der gleitenden Fenster), wobei kurzzeitige Abfälle bis rund \(3{,}15\,\text{Hz}\) zeitlich mit Bodenwellen und Steigungen zusammenfallen. Diese Effekte sind plausibel auf temporäre Punktwolkenverzerrungen zurückzuführen, welche die Stabilität der Clusterbildung beeinflussen. Der schnelle Wiederanstieg der Frequenz zeigt jedoch, dass die Trackingkomponente insgesamt reibungsfrei weiterarbeiten kann und keine nachhaltige Instabilität entsteht.

Die mit \texttt{resource\_probe.py} erhobenen Ressourcenwerte liegen im moderaten Bereich: Über die Versuchszeit von \(180\,\text{s}\) wurde eine mittlere CPU-Auslastung der Verarbeitungskette von \(\approx 15{,}2\,\%\) bei einem Gesamtspeicherbedarf von rund \(\approx 202\,\text{MB}\) gemessen. Damit verbleiben ausreichende Reserven, um auch unter dynamischen Fahrzuständen Lastspitzen abzufangen.

Insgesamt bestätigt dieses Szenario, dass die Verarbeitungskette auch unter erhöhten dynamischen Anforderungen eine konsistente Boxenbildung und weitgehend stabile ID-Kontinuität ermöglicht.


\subsubsection{Urbanes Gebiet bei 50\,km/h (Unebene Fahrbahn)}
\label{sec:urban_50kmh}

Der Test im urbanen Umfeld bei etwa \(50\,\text{km/h}\) diente der Überprüfung der Robustheit der Bodensegmentierung unter realistischen Bedingungen mit Fahrbahnunebenheiten. Die Messung der Ressourcen mittels \texttt{resource\_probe.py} zeigte eine sehr geringe Gesamtauslastung der Verarbeitungskette: Die CPU-Last aller Knoten lag im Mittel bei lediglich \(11{,}7\,\%\), der gesamte Speicherbedarf bei rund \(200\,\text{MB}\). Einzelne Knoten wie \texttt{crop\_box\_node} oder \texttt{cluster\_extraction\_node} erreichten jeweils nur wenige Prozentpunkte Auslastung, sodass ausreichende Leistungsreserven bestehen.

Die Latenzmessungen mit \texttt{latency\_probe.py} verdeutlichen, dass die Ende-zu-Ende-Verarbeitung selbst in diesem dynamischen Szenario stabil bleibt. Die Verzögerung von \texttt{crop\_box} bis \texttt{tracking} bewegte sich im stationären Zustand im Bereich von etwa \(20\)–\(26\,\text{ms}\). Zwar steigen die Latenzen einzelner Schritte wie \texttt{ransac\_ground} oder \texttt{cluster\_extraction}, wenn durch unebene Fahrbahn oder zusätzliche Objekte mehr Punktdaten verarbeitet werden müssen, jedoch bleiben diese Effekte moderat und beeinträchtigen die Gesamtlatenz nicht kritisch.

Auch die Topic-Frequenzen zeigten ein stabiles Verhalten. Die Eingangsfrequenz des Sensors \texttt{/ouster/points} pendelte sich auf etwa \(3{,}2\,\text{Hz}\) ein, wobei die Standardabweichung im Zeitverlauf abnahm und somit eine zunehmende Stabilität der Datenrate belegte. Die Ausgangsfrequenz der Tracks (\texttt{/bench/tracks\_raw}) lag im Mittel bei rund \(3{,}0\,\text{Hz}\). Größere maximale Intervalle sind hier durch das Publikationsverhalten bedingt: \texttt{tracks\_raw} sendet nur dann Nachrichten, wenn tatsächlich verfolgte Objekte vorhanden sind, sodass Phasen ohne erkannte Ziele als lange Intervalle in der Frequenzmessung erscheinen.

Insgesamt zeigt dieses Szenario, dass die Bodensegmentierung und die gesamte Verarbeitungskette auch bei Fahrbahnunebenheiten und im urbanen Umfeld robust und performant arbeiten. Weder die Latenzen noch die Datenraten weisen kritische Schwankungen auf, sodass die Anforderungen an eine stabile, echtzeitfähige Verarbeitung erfüllt werden.


\begin{table}[H]
  \centering
  \footnotesize
  \setlength{\tabcolsep}{4pt}
  \renewcommand{\arraystretch}{0.95}
  \begin{tabular}{|p{3.4cm}|p{3.1cm}|p{2.1cm}|p{2.1cm}|p{2.1cm}|}
    \hline
    \textbf{Szenario} & \textbf{CPU / RSS} & \textbf{mittlere Latenz} & \textbf{f\_in} & \textbf{f\_out} \\
    \hline

    Gerade Strecke, Stand &
    32\,\% / 197\,MB &
    86\,ms (stabil 81\,ms) &
    2.5\,Hz &
    1.8\,Hz \\
    \hline

    Kurve, 40\,km/h (Urban) &
    31.4\,\% / 198\,MB &
    39\,ms &
    3.7\,Hz &
    1.8\,Hz \\
    \hline

    Innenstadt + Steigung, 30\,km/h &
    12.3\,\% / 200\,MB &
    30\,ms &
    3.3--3.4\,Hz &
    2\,Hz \\
    \hline

    Kreuzung, 50\,km/h &
    n.\,a. &
    21--24\,ms &
    3.21--3.24\,Hz &
    kurze Einbrüche bis \(\approx 3{,}15\,\text{Hz}\)  \\
    \hline

    Urban, 50\,km/h (Unebene Fläche) &
    11.7\,\% / 200\,MB &
    20--26\,ms &
    3.2\,Hz &
    3.0\,Hz \\
    \hline
  \end{tabular}
  \caption{Vergleichende Übersicht der Messgrößen je Szenario}
  \label{tab:szenario_ueberblick}
\end{table}

\section{Bewertung des Algorithmus}
\label{sec:bewertung_algorithmus}

Die Auswertung der Versuchsreihen zeigt, dass der entwickelte Pipeline-Algorithmus die Anforderungen an Echtzeitfähigkeit, Ressourceneffizienz und Robustheit größtenteils erfüllt. Die wichtigsten Erkenntnisse lassen sich wie folgt zusammenfassen:
\begin{itemize}
  \item \textbf{Echtzeitfähigkeit:} In allen Szenarien bleiben die End-to-End-Latenzen deutlich unter der Zielmarke von \(100\,\mathrm{ms}\); vielfach liegen sie sogar im Bereich von \(20{-}40\,\mathrm{ms}\). Damit reagiert das System ausreichend schnell, um in typischen Fahrgeschwindigkeiten sichere Entscheidungen zu ermöglichen.
  \item \textbf{Ressourceneffizienz:} Die CPU-Auslastung bleibt durchweg unter \(35\,\%\), der Speicherbedarf stabil um \(200\,\mathrm{MB}\). Somit existiert genügend Spielraum für zusätzliche Verarbeitungsschritte oder parallele Aufgaben auf der Zielhardware.
  \item \textbf{Stabilität der Datenraten:} Die gemessenen Frequenzen der Punktwolken- und Track-Topics zeigen eine überwiegend konstante Verarbeitung ohne langanhaltende Aussetzer. Kurzzeitige Einbrüche (z.\,B. durch Bodenwellen) werden schnell kompensiert, was auf eine robuste Pipeline hinweist.
  \item \textbf{Tracking-Qualität:} In statischen Szenarien treten relativ viele Track-Geburten mit kurzen Lebensdauern auf. Dies deutet auf Optimierungspotenzial bei den Schwellenwerten der Track-Verwaltung hin, beeinträchtigt aber die grundlegende Leistungsfähigkeit nicht.
  \item \textbf{Übertragbarkeit:} Die ähnlichen Ressourcen- und Latenzprofile in heterogenen Szenarien (Stand, urbanes Fahren, Steigung, Unebenheiten) sprechen dafür, dass der Algorithmus gut skalierbar und gegenüber Umgebungsvariationen robust ist.
\end{itemize}

In Summe erfüllt die aktuelle Konfiguration der Verarbeitungskette die definierten Echtzeit- und Stabilitätsanforderungen und bietet gleichzeitig Ansatzpunkte für weitere Verbesserungen, insbesondere im Feintuning der Tracking-Parameter zur Reduktion unnötiger ID-Wechsel.
Damit sind zentrale Kriterien aus den in Tabelle~\ref{tab:anforderungen_messkette} formulierten Anforderungen an den Algorithmus erfüllt, namentlich die Einhaltung der End-to-End-Latenz \(<100\,\mathrm{ms}\) sowie eine ressourcenschonende Ausführung auf Standardhardware.
